\documentclass[fleqn,12pt]{article}

\usepackage[utf8]{inputenc}
\usepackage[T1]{fontenc}
\usepackage{mathtools} %loads amsmath
\usepackage{amsthm}
\usepackage{amsfonts}
\usepackage{amsmath}
\usepackage{amssymb}
\usepackage{graphicx}
\usepackage{tikz}
\usetikzlibrary{arrows,automata,positioning}
\usepackage{arydshln}
\usepackage{stmaryrd}
\usepackage{pdflscape}
\usepackage{rotating}
\usepackage[margin=2cm]{geometry}
\usepackage[font=small,labelfont=bf,tableposition=top]{caption}
\usepackage{ebproof}

\tikzset{initial text={}}

\usepackage{fancyhdr}
\setlength{\headheight}{26pt}
\pagestyle{fancy}
\lhead{Static Program Analysis WS 2016/17 -- Series 08 \\ \small{Igor Dudschenko (296764), Oliver Westphal (358321)}}
\rhead{}

\newcommand\dbrackets[1]{\llbracket #1 \rrbracket}

\setlength{\parindent}{0cm}
\newcommand\note[1]{\textcolor{red}{#1}}

\begin{document}
\section*{Exercise 1}
\subsection*{a)}
To show:
$$\alpha(\bot_L) = \bot_M$$
Proof:
$$\alpha \text{ is uniquely dermined by $\gamma$ as follows: } \alpha(l) = \bigsqcap \{m \in M | l \sqsubseteq_L \gamma(m)\}$$
$$\Rightarrow  \alpha(\bot_L) = \bigsqcap \{m \in M | \bot_L \sqsubseteq_L \gamma(m) \}\}$$
$$\Rightarrow \alpha(\bot_L) = \bot_M$$

\subsection*{b)}
To show:
$$\gamma(\bot_M) = \bot_L$$
Proof:
$$\gamma \text{ is uniquely dermined by $\alpha$ as follows: } \gamma(m) = \bigsqcup \{l \in L | \alpha(l) \sqsubseteq_M m\}$$
$$\Rightarrow \gamma(\bot_M) = \bigsqcup \{l \in L | \alpha(l) \sqsubseteq_M \bot_M\} $$
$$\Rightarrow \gamma(\bot_M) = \bot_L$$

\subsection*{c)}
To show:
$$\alpha \circ \gamma \circ \alpha = \alpha$$
Proof:
$$\forall l \in L: l \sqsubseteq_{L} \gamma(\alpha(l)) \text{ and: } \forall m \in M: \alpha(\gamma(m)) \sqsubseteq_{M} m$$
$$\Rightarrow \alpha \sqsubseteq_M \alpha \circ (\gamma \circ \alpha) \text{ (monotonicity of $\alpha$)}$$
And:
$$\Rightarrow (\alpha \circ \gamma) \circ \alpha \sqsubseteq_M \alpha \text{ (monotonicity of $\alpha$)}$$
Thus:
$$\alpha \circ \gamma \circ \alpha = \alpha$$
\subsection*{d)}
To show:
$$\gamma \circ \alpha \circ \gamma = \gamma$$
Proof:
$$\forall l \in L: l \sqsubseteq_{L} \gamma(\alpha(l)) \text{ and: } \forall m \in M: \alpha(\gamma(m)) \sqsubseteq_{M} m$$
$$\Rightarrow \gamma \sqsubseteq_L (\gamma \circ \alpha) \circ \gamma \text{ (monotonicity of $\gamma$)}$$
And:
$$\Rightarrow \gamma \circ (\alpha \circ \gamma) \sqsubseteq_L \gamma \text{ (monotonicity of $\gamma$)}$$
Thus:
$$\gamma \circ \alpha \circ \gamma = \gamma$$
\section*{Exercise 2}

\subsection*{1)}
Extend the execution function introduced in the lecture by:\\
(push)
\begin{prooftree}
	\Infer0[]{}
	\Infer1[]{<s.push(a),\sigma> \rightarrow \sigma [s \mapsto val_{\sigma}(a) \cdot val_{\sigma}(s) ]}
\end{prooftree}

(pop)
\begin{prooftree}
	\Infer0[]{}
	\Infer1[]{<s.pop(),\sigma> \rightarrow \sigma [s \mapsto val_{\sigma}(s) [2,length(val_{\sigma}(s))] ]}
\end{prooftree}
\subsection*{2)}
Given:
\begin{enumerate}
	\item $L = 2^{\mathbb{Z^{*}} \times \mathbb{Z^{*}}}$
	\item $M_1 = 2^{\{H,S\}}$
	\item $M_2 = 2^{\mathbb{Z}}$
	\item $(\alpha_2,\gamma_2) \text{ with } \mathbb{Z}^* \times \mathbb{Z}^* \rightarrow \mathbb{Z}:(a,b)\mapsto length(a) - length(b)$
	\item $M_1 \times M_2 = 2^{\{H,S\}} \times 2^{\mathbb{Z}}$
\end{enumerate}
Required for $(\alpha,\gamma)$ is a product of:
	\begin{enumerate}
		\item $(\alpha_1,\gamma_1)$ between $L$ and $M_1$
		\item $(\alpha_2,\gamma_2)$ between $L$ and $M_2$
	\end{enumerate}
Construct $(\alpha_1,\gamma_1)$:
  \[
    \alpha_1(m,n) =
    \begin{cases}
         H & \text{, for } \exists a_{top}  \in \mathbb{Z} \text{ and }x,y\in \mathbb{Z}^*: m = a_{top} \circ x \land n = a_{top} \circ y \\
         S & \text{, for } \exists a  \in \mathbb{Z}^*: m = x \circ n \lor n = x \circ m \\
    \end{cases}
  \]
  Note: $a_{top}$ indicates that this is the same top element for both stacks, therefore use H. For S we check if either n can be extended by any substack x, s.t. $m=x \circ n$ or vice versa with m, to check for possible suffixes.  
\section*{Exercise 3}
\subsection*{a)}

\subsection*{b)}

\subsection*{c)}

\subsection*{d)}

\end{document}
