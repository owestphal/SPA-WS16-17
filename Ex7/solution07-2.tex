\documentclass[fleqn,12pt]{article}

\usepackage[utf8]{inputenc}
\usepackage[T1]{fontenc}
\usepackage{mathtools} %loads amsmath
\usepackage{amsthm}
\usepackage{amsfonts}
\usepackage{amsmath}
\usepackage{amssymb}
\usepackage{graphicx}
\usepackage{tikz}
\usetikzlibrary{arrows,automata,positioning}
\usepackage{arydshln}
\usepackage{stmaryrd}
\usepackage{pdflscape}
\usepackage{rotating}
\usepackage[margin=2cm]{geometry}
\usepackage[font=small,labelfont=bf,tableposition=top]{caption}

\tikzset{initial text={}}

\usepackage{fancyhdr}
\setlength{\headheight}{26pt}
\pagestyle{fancy}
\lhead{Static Program Analysis WS 2016/17 -- Series 07 \\ \small{Igor Dudschenko (296764), Oliver Westphal (358321)}}
\rhead{}

\newcommand\dbrackets[1]{\llbracket #1 \rrbracket}

\setlength{\parindent}{0cm}
\newcommand\note[1]{\textcolor{red}{#1}}

\begin{document}
\section*{Exercise 1}
\subsection*{a)}
to show: $(L_1, \alpha_2 \circ \alpha_1, \gamma_1 \circ \gamma_2, M_2)$ is Galois-Connection, i.e.
$$\forall l \in L_1 : l \sqsubseteq_{L_1} \gamma_1 \circ \gamma_2 (\alpha_2 \circ \alpha_1 (l))$$
$$\forall m \in M_2 : \alpha_2 \circ \alpha_1(\gamma_1 \circ \gamma_2  (m)) \sqsubseteq_{M_2} m$$\\
Since $\alpha_1(l) \in L_2$ for $l \in L_1$ from the def. of Galois-Connections we get $\alpha_1(l) \sqsubseteq_{L_2} \gamma_2(\alpha_2(\alpha_1(l))$ because $\gamma_1$ is monotonic we also have $\gamma_1(\alpha_1(l)) \sqsubseteq_{L_2} \gamma_1(\gamma_2(\alpha_2(\alpha_1(l)))$ combining this with $l \sqsubseteq_{L_1} \gamma_1(\alpha_1(l))$ from the Galois-def. yields $l \sqsubseteq_{L_1} \gamma_1(\alpha_1(l)) \sqsubseteq_{L_2} \gamma_1(\gamma_2(\alpha_2(\alpha_1(l)))$ and therefore the first part of the claim.\\
\\
Analogously we show the second part:\\
$\gamma_2(m)\in M_1$ for $m\in M_2 \Rightarrow \alpha_1(\gamma_1(\gamma_2(m)) \sqsubseteq_{M_1} \gamma_2(m)$, and  again using monotonicity and the Galois property we get $\alpha_2(\alpha_1(\gamma_1(\gamma_2(m))) \sqsubseteq_{M_2} \alpha_2(\gamma_2(m)) \sqsubseteq_{M_2} m \qed$
\subsection*{b)}

\section*{Exercise 2}
\subsection*{a)}
For $Z \in 2^\mathbb{Z}$, $\gamma(\alpha(Z) = \gamma(\{ f(z) | z \in Z \}) = \bigcup_{p\in \{ f(z) | z \in Z \}} \{p n | n \in \mathbb{Z}_{>0}\} = {f(z) | z \in Z, n \in \mathbb{Z}_{>0}} \sqsupseteq_L \{f(z) n_z | z \in Z\} $ where $\sqsubseteq_L = \subseteq$ and
$n_z = 
	\begin{cases}
	|z| \text{ if z odd}\\
	\frac{|z|}{2} \text{ otherwise}	
	\end{cases}$\\
Then $\gamma(\alpha(Z) \sqsupseteq_L \{f(z) n_z | z \in Z\} = Z$.\\
\\
For $P \in 2^{\{0,1,2,-1,-2\}}$, $\alpha(\gamma(P)) = \alpha(\bigcup_{p\in P} \{p n | n \in \mathbb{Z}_{>0}\}) = \{f(z) | z \in \bigcup_{p\in P} \{p n | n \in \mathbb{Z}_{>0}\} \sqsubseteq_M P $ where $\sqsubseteq_M = \supseteq$.\\
\\
So $(2^\mathbb{Z},\alpha,\gamma,2^{\{0,1,2,-1,-2\}})$ is Galois connection, with the lattices $(2^\mathbb{Z},\subseteq)$ and $(2^{\{0,1,2,-1,-2\}},\supseteq)$
\subsection*{b)}

\subsection*{c)}

\subsection*{d)}

\subsection*{e)}

\end{document}
