\documentclass[fleqn,12pt]{article}

\usepackage[utf8]{inputenc}
\usepackage[T1]{fontenc}
\usepackage{mathtools} %loads amsmath
\usepackage{amsthm}
\usepackage{amsfonts}
\usepackage{amsmath}
\usepackage{amssymb}
\usepackage{graphicx}
\usepackage{tikz}
\usetikzlibrary{arrows,automata,positioning}
\usepackage{arydshln}
\usepackage{stmaryrd}
\usepackage{pdflscape}
\usepackage{rotating}
\usepackage[margin=2cm]{geometry}
\usepackage[font=small,labelfont=bf,tableposition=top]{caption}
\usepackage{ebproof}

\tikzset{initial text={}}

\usepackage{fancyhdr}
\setlength{\headheight}{26pt}
\pagestyle{fancy}
\lhead{Static Program Analysis WS 2016/17 -- Series 12 \\ \small{Igor Dudschenko (296764), Oliver Westphal (358321)}}
\rhead{}

\newcommand\dbrackets[1]{\llbracket #1 \rrbracket}

\setlength{\parindent}{0cm}
\newcommand\note[1]{\textcolor{red}{#1}}

\begin{document}
\section*{Exercise 1}
$S:=(Lab,E,F,(D,\sqsubseteq),\iota,\varphi)$ is determined by:
\begin{itemize}
\item $D:=\{\delta | \delta: Var_c \rightarrow \mathbb{Z} \cup \{\bot, \top\}\}$
\item $\bot \sqsubseteq z \sqsubseteq \top$ for every $z \in \mathbb{Z}$
\item $\iota := \delta_{\top} \in D$
\item For each $l \in Lab \\ \{l_c,l_n,l_x,l_r | (l_c,l_r,l_x,l_r) \in iflow\}$,
\[
 \varphi_l(\delta) := 
  \begin{cases} 
   \delta & \text{if } B^{l} = skip \text{ or } B^{l} \in BExp \\
   \delta [x \mapsto val_{\delta}(a)]       & \text{if } B^l = (x:=a)
  \end{cases}
\]
where
$$val_{\delta}(x):=\delta(x), val_{\delta}(z):=[z,z]$$
$$val_{\delta}(a_1+a_2):=val_{\delta}(a_1) \oplus val_{\delta}(a_2)$$
$$val_{\delta}(a_1-a_2):=val_{\delta}(a_1) \ominus val_{\delta}(a_2)$$
$$val_{\delta}(a_1*a_2):=val_{\delta}(a_1) \odot val_{\delta}(a_2)$$
\item Whenever $p$ $c$ contains $[call \text{ } P(a,z)]^{l_c}_{l_r}$ and $[proc \text{ } P(val \text{ } x,res \text{ } y)]^{l_n}$ is $c \text{ } [end]^{l_x}$,
\begin{itemize}
\item call/entry: set input and reset output intervals
$$\varphi_{l_c}(\delta) := \delta [x \mapsto val_{\delta}(a),y \mapsto \top], \varphi_{l_n}(\delta) := \delta$$
\item exit/return: reset intervals (x,y possible used in calling context) and set interval according to return value
$$\varphi_{l_x}(\delta) := \delta, \varphi_{l_r}(\delta ', \delta) := \delta '[x \mapsto \delta(x), y \mapsto \delta(y),z\mapsto \delta '(y)] $$
\end{itemize}
\end{itemize}
\section*{Exercise 2}
\subsection*{a)}
$$mop(S) \sqsubseteq mvp(S)$$
This cannot hold in the interprocedural case, since $Path(l) \nsubseteq VPath(l)$ for every $l \in Lab$.
\subsection*{b)}
$$mvp(S) \sqsubseteq mop(S)$$
This holds, since $VPath(l) \subseteq Path(l)$ for every $l \in Lab$.
\subsection*{c)}
$$mop(S) \sqsubseteq fix(S)$$
This holds, since from a), b) and e) we know: $mop(S) \not\sqsubseteq mvp(S)$, $mvp(S) \sqsubseteq mop(S)$ and $mvp(S) \sqsubseteq fix(S)$, therefore $mvp(S) \sqsubseteq mop(S) \sqsubseteq fix(S)$.
\subsection*{d)}
$$fix(S) \sqsubseteq mop(S)$$
If this property holds, than $fix(S) \sqsubseteq mop(S)$ and (as we showed in c)) $mop(S) \sqsubseteq fix(S)$ hold. But this would mean that $fix(S) = mop(S)$, which cannot be true since we already know from the lecture (Example 6.2) that this property doesn't hold in the intraprocedural part.
\subsection*{e)}
$$mvp(S) \sqsubseteq fix(S)$$
This holds trivially from Theorem 20.2.
Also: we know from b) and c): $mvp(S) \sqsubseteq mop(S)$ and $mop(S) \sqsubseteq fix(S)$, therefore $mvp(S) \sqsubseteq mop(S) \sqsubseteq fix(S)$.
\subsection*{f)}
$$fix(S) \sqsubseteq mvp(S)$$
This property only holds if all $\varphi_l$ are distributive, which is generally not given in S.
\end{document}
