\documentclass[fleqn,12pt]{article}

\usepackage[utf8]{inputenc}
\usepackage[T1]{fontenc}
\usepackage{mathtools} %loads amsmath
\usepackage{amsthm}
\usepackage{amsfonts}
\usepackage{amsmath}
\usepackage{amssymb}
\usepackage{graphicx}
\usepackage{tikz}
\usetikzlibrary{arrows,automata,positioning}
\usepackage{arydshln}
\usepackage{stmaryrd}
\usepackage{pdflscape}
\usepackage{rotating}
\usepackage[margin=2cm]{geometry}
\usepackage[font=small,labelfont=bf,tableposition=top]{caption}
\usepackage{ebproof}

\tikzset{initial text={}}

\usepackage{fancyhdr}
\setlength{\headheight}{26pt}
\pagestyle{fancy}
\lhead{Static Program Analysis WS 2016/17 -- Series 11 \\ \small{Igor Dudschenko (296764), Oliver Westphal (358321)}}
\rhead{}

\newcommand\dbrackets[1]{\llbracket #1 \rrbracket}

\setlength{\parindent}{0cm}
\newcommand\note[1]{\textcolor{red}{#1}}

\begin{document}
\section*{Exercise 1}
$$p_1 = x>y \text{, }p_2 = x>3 \text{, }p_3 = y \leq 2 \text{, } p_4 = x = 2$$
\subsection*{a)}
$$Q_1 = \{p_1\}\text{, } Q_2 =  \{p_3\}$$
$$Q_1 \sqcap Q_2 = \overline{Q_1 \wedge Q_2} = p_1 \wedge p_2 \wedge p_3 \wedge \lnot p_4$$
$$Q_1 \sqcup Q_2 = \overline{Q_1 \vee Q_2} = p_1 \wedge p_2 \wedge p_3 \wedge \lnot p_4$$
\subsection*{b)}
$$Q_1 = \{p_1,p_2\}\text{, } Q_2 =  \{p_2,p_4\}$$
$$Q_1 \sqcap Q_2 = \overline{Q_1 \wedge Q_2} = false$$
$$Q_1 \sqcup Q_2 = \overline{Q_1 \vee Q_2} = \overline{(p_1 \wedge p_2) \vee (p_2 \wedge p_4)} = \overline{p_2 \wedge (p_1 \vee p_4)} = p_2$$
\subsection*{c)}
$$Q_1 = \{p_1,\lnot p_2\}\text{, } Q_2 =  \{\lnot p_3\}$$
$$Q_1 \sqcap Q_2 = \overline{Q_1 \wedge Q_2} = p_1 \wedge \lnot p_2 \wedge \lnot p_3$$
$$Q_1 \sqcup Q_2 = \overline{Q_1 \vee Q_2} = \overline{(p_1 \wedge \lnot p_2) \vee \lnot p_3} = \overline{(p_1 \vee \lnot p_3) \wedge (\lnot p2 \vee \lnot p_3)} = \lnot p_3$$
\section*{Exercise 2}

\subsection*{a)}
$$P=\emptyset \Rightarrow Abs(P) = \{true,false\}$$
\subsection*{b)}
$$\langle 0, true \rangle \Rightarrow \langle 1, true \rangle \Rightarrow \langle 2, true \rangle \Rightarrow \langle 3, true \rangle \Rightarrow \langle 4, true \rangle \Rightarrow \langle 51, true \rangle$$
\subsection*{c)}
\begin{itemize}
\item $b_0 := true$
\item (assgn) $b_1 := \exists y'.(b_0[y\mapsto y'] \wedge y=b[y \mapsto y']) \equiv (y=b) $
\item (if1) $b_2 := (b_1 \wedge b > 0) \equiv (y=b \wedge b>0)$
\item (assgn) $b_3 := \exists y'. (b_2[y \mapsto y'] \wedge y := y -1 [y \mapsto y']) =  \exists y'. (y'=b \wedge y = y' - 1 \wedge b>0) \equiv (y=b - 1 \wedge b>0)$
\item (if1) $b_4 := (b_3 \wedge y < 0) \equiv (y=b-1 \wedge b>0 \wedge y<0) \equiv false(\text{since: }y \in \mathbb{N})$
\end{itemize}
\subsection*{d)}
$$p_0 := (y = b)$$
$$p_1 := (y = b - 1)$$
$$p_2 := (b>0)$$
$$p_2 := (y<0)$$
$$\langle 0, true \rangle \Rightarrow \langle 1, p_0 \wedge \lnot p_1 \rangle \Rightarrow \langle 2, p_0 \wedge \lnot p_1 \wedge p_2 \rangle \Rightarrow \langle 3, \lnot p_0 \wedge p_1 \wedge p_2 \rangle \Rightarrow \langle 4, \lnot p_0 \wedge p_1 \wedge p_2 \wedge p_3 \rangle$$
\subsection*{e)}
In the example shown in d one can see that one step suffices, since $ \lnot p_0 \wedge p_1 \wedge p_2 \wedge p_3 \equiv false$ 
\end{document}
