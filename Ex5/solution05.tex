\documentclass[fleqn,12pt]{article}

\usepackage[utf8]{inputenc}
\usepackage[T1]{fontenc}
\usepackage{mathtools} %loads amsmath
\usepackage{amsthm}
\usepackage{amsfonts}
\usepackage{amsmath}
\usepackage{amssymb}
\usepackage{graphicx}
\usepackage{tikz}
\usetikzlibrary{arrows,automata,positioning}
\usepackage{arydshln}
\usepackage{stmaryrd}
\usetikzlibrary{arrows,automata}
\usepackage{ stmaryrd }

\usepackage{fancyhdr}
\setlength{\headheight}{26pt}
\pagestyle{fancy}
\lhead{Static Program Analysis WS 2016/17 -- Series 05 \\ \small{Igor Dudschenko (296764), Oliver Westphal (358321)}}
\rhead{}

\newcommand\dbrackets[1]{\llbracket #1 \rrbracket}

\setlength{\parindent}{0cm}
\newcommand\note[1]{\textcolor{red}{#1}}

\begin{document}
\section*{Exercise 1}
\subsection*{a)}
\[ \varphi_l(\delta) :=
  \begin{cases}
    \delta       & \quad \text{if } B^l = skip \text{or} B^l \in BExp\\
    \delta [x] \mapsto val_d(a) & \quad \text{if } B^l=(x:=a)\\
  \end{cases}
\]
$$val_{\delta}(x):=\delta (x)$$
$$val_{\delta}(z):=[z,z]$$
$$val_{\delta}(a_1/a_2):=val_{\delta}(a_1) \oslash val_{\delta}(a_2)$$
$$[y_1,y_2] \oslash [z_1,z_2]=[\bigsqcup\{y_1/z_1,y_1/z_2,y_2/z_1,y_2/z_2\},\bigsqcap\{y_1/z_1,y_1/z_2,y_2/z_1,y_2/z_2\}]$$
\subsection*{b)}
To show:
$$\text{If: }x_1 \sqsubseteq x_2 \text{ than also } \delta(x_1) \sqsubseteq \delta(x_2)$$
For:
\[ \varphi_l(\delta) :=
  \begin{cases}
    \delta       & \quad \text{if } B^l = skip \text{or} B^l \in BExp\\
  \end{cases}
\]
the statement obviously holds.
We have four cases to show for:
\[ \varphi_l(\delta) :=
  \begin{cases}
    \delta [x] \mapsto val_d(a) & \quad \text{if } B^l=(x:=a)\\
  \end{cases}
\]
\begin{itemize}
	\item{$\oplus$ :} $[y_1,y_2] \oplus [z_1,z_2] := [y_1+z_1,y_2+z_2]$
	\item{$\ominus$ :} $[y_1,y_2] \ominus [z_1,z_2] := [y_1-z_2,y_2-z_1]$
	\item{$\otimes$ :} $[y_1,y_2] \otimes [z_1,z_2]=[\bigsqcap\{y_1/z_1,y_1/z_2,y_2/z_1,y_2/z_2\},\bigsqcup\{y_1/z_1,y_1/z_2,y_2/z_1,y_2/z_2\}]$
	\item{$\oslash$ :} $[y_1,y_2] \oslash [z_1,z_2]=[\bigsqcup\{y_1/z_1,y_1/z_2,y_2/z_1,y_2/z_2\},\bigsqcap\{y_1/z_1,y_1/z_2,y_2/z_1,y_2/z_2\}]$
\end{itemize}

\section*{Exercise 2}

\section*{Exercise 3}
\subsection*{a)}
We define $\sqsubseteq$ as the smallest relation such that\\
\begin{enumerate}
\item $d \sqsubseteq d$, $\forall d \in D$
\item $(n,0) \sqsubseteq \infty$, $\forall n \in \mathbb{N}$
\item $\infty \sqsubseteq (n,1)$, $\forall n \in \mathbb{N}$
\item $(n,s) \sqsubseteq (n',s')$ iff \\ 
	$s < s'$\\
	or $s=s'=0 \wedge n \leq n'$\\
	or $s=s'=1 \wedge n \geq n'$	
\end{enumerate}
Then $(D\sqsubseteq)$ is a complete lattice with $\bot=(0,0)$ and $\top=(0,1)$.
This lattice contains both infinite ascending and descending chains as can be seen by "visualizing" $\sqsubseteq$:\\
$(0,0)\sqsubseteq(1,0)\sqsubseteq(2,0)\sqsubseteq \underbrace{\dots}_\text{inf. asc.} \sqsubseteq\infty\sqsubseteq \underbrace{\dots}_\text{inf. desc.} \sqsubseteq (2,1) \sqsubseteq (1,1) \sqsubseteq (0,1)$
\newpage
\subsection*{b)}
Let $p_n$ denote the n-th prime number with $p_1=2, p_2=3, \dots$.
Also we define $P_n$ to be the set of all multiples of $p_n$, i.e. $$P_n := \{p_n^k | k \in \mathbb{N}_+\}$$\\
e.g. $P_2 = \{2,4,6,8,\dots \}$.\\
Clearly each of these sets is infinite and $P_n \cap P_m = \emptyset$ for $n\neq m$. Also $0,1 \not \in P_n \ \forall n \in \mathbb{N}_+$.\\
Now we define $\preceq$ as the transitive-reflexive closure of $\preceq'$, with $\preceq'$ defined as follows:\\
$\forall n,k\in \mathbb{N}_+$
\begin{itemize}
\item $(0,0) \preceq' (1,0)$
\item $(1,0) \preceq' (p_n,0)$
\item $(p_n^k,0) \preceq' (p_n^{k+1},0)$
\item $(p_n^k,0) \preceq' \infty$

\item $\infty \preceq' (p_n^k,1)$
\item $(p_n^{k+1},1) \preceq' (p_n^k,1)$
\item $(p_n,1) \preceq' (1,1)$
\item $(1,1) \preceq' (0,1)$
\end{itemize}
Then $(D,\preceq)$ is a complete lattice ($\bot=(0,0),\top=(0,1)$) with both infinitely many
pairwise disjoint infinite ascending and infinitely many pairwise disjoint infinite descending chains.
Since every set $P_n \times \{0\}$ is a infinite ascending chain and each $P_n \times \{1\}$ is a infinite descending chain (see picture).
%\includegraphics[scale=.4]{Ex3b.pdf}
\subsection*{c)}
for $(D,\sqsubseteq)$:\\
$d_1 \sqcup d2 =
\begin{cases}
d_1 \text{, if } d_2 \sqsubseteq d_1 \\
d_2 \text{, otherwise}
\end{cases}$\\\\
Note that $\sqsubseteq$ is total.\\\\
$d_1 \triangledown d2 =
\begin{cases}
\infty \text{, if } d_1,d_2 \in \mathbb{N} \times \{0\}, d_1 \neq d_2 \\
d_1 \sqcup d_2 \text{, otherwise}
\end{cases}$\\\\

for $(D,\preceq)$:\\
$d_1 \sqcup d2 =
\begin{cases}
d_1 \text{, if } d_2 \sqsubseteq d_1 \\
d_2 \text{, if } d_2 \sqsubseteq d_1 \\
\infty \text{, if } d_1,d_2 \in \mathbb{N}\times \{0\}, d_1 \not\preceq d_2, d_2 \not\preceq d_1 \\
(1,1) \text{, if } d_1,d_2 \in \mathbb{N}\times \{1\}, d_1 \not\preceq d_2, d_2 \not\preceq d_1
\end{cases}$\\\\

$d_1 \triangledown d2 =
\begin{cases}
\infty \text{, if } d_1 \in P_n \times \{0\}, d_2 \in P_m \times \{0\}, n \neq m\\
d_1 \sqcup d_2 \text{, otherwise}
\end{cases}$

\section*{Exercise 4}

\end{document}
