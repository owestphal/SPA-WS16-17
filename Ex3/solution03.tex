\documentclass[fleqn,12pt]{article}

\usepackage[utf8]{inputenc}
\usepackage[T1]{fontenc}
\usepackage{mathtools} %loads amsmath
\usepackage{amsthm}
\usepackage{amsfonts}
\usepackage{amsmath}
\usepackage{amssymb}
\usepackage{graphicx}
\usepackage{tikz}
\usetikzlibrary{arrows,automata,positioning}
\usepackage{arydshln}
\usepackage{stmaryrd}
\usetikzlibrary{arrows,automata}
\usepackage{ stmaryrd }

\usepackage{fancyhdr}
\setlength{\headheight}{26pt}
\pagestyle{fancy}
\lhead{Static Program Analysis WS 2016/17 -- Series 03 \\ \small{Igor Dudschenko (296764), Oliver Westphal (358321), Mitja Schmakeit (357076)}}
\rhead{}

\newcommand\dbrackets[1]{\llbracket #1 \rrbracket}

\setlength{\parindent}{0cm}
\newcommand\note[1]{\textcolor{red}{#1}}

\begin{document}
\section*{Exercise 1}
\subsection*{a)}
g has two fixpoints: 0 and 1, where 0 is the least FP.
\subsection*{b)}
f has the set of fixpoints: $\{2\Pi i| i \in \mathbb{N}\}$, where 0 is the least FP.
\subsection*{c)}
f has the fixpoints: $\emptyset, \{0\}, \mathbb{N}_0$, with $\emptyset$ being the least FP.
\subsection*{d)}
f has the set of fixpoints $\{-1, 0, 1\}$, with $-1$ being the least FP. 

\newcommand\lfp{\text{lfp}}
\section*{Exercise 2}
\subsection*{a)}
Consider the complete lattice $(D,\leq)$ with $D = \{0,1,..,10\}$\\
and the monotonic function $f(x) := x$\\
Then, $C := D \setminus \{0\}$ is closed and $f(x) \in C$ for all $x \in C$.
But $\lfp(f) = 0$, therefore $\lfp(f) \not\in C$.\\
Thus the claim does not hold.
\subsection*{b)}
\newcommand\sqle\sqsubseteq
Let $\bot$ be the least element of $(D, \sqle)$, then $\bot \sqle x$ and by monotonocity of $f$
\begin{align}
	\label{eq:L2b1}
	f^n(\bot) \sqle f^n(x)
\end{align}
holds for all $n \in \mathbb{N}$. From the assumption and monotonicity of $f$ it follows that
\begin{align*}
	f^n(x) \sqle f^{n-1}(x) \sqle \dots \sqle f(x) \sqle x
\end{align*}
and by transitivity of $\sqle$
\begin{align}
	\label{eq:L2b2}
	f^n(x) \sqle x \text{ for all } n \in \mathbb N
\end{align}
From equations \ref{eq:L2b1} and \ref{eq:L2b2} and transitivity we get
\begin{align}
	f^n(\bot) \sqle x \text{ for all } n \in \mathbb N
\end{align}
Since $\lfp(f)=\bigsqcup\{f^k(\bot) | k \in \mathbb N\}$ (Theorem 4.5), we can choose $i \in \mathbb N$ s.t. $\lfp(f) = f^i(\bot)$
\begin{align*}
\implies \lfp(f)=f^i(\bot) \sqle x
\end{align*}
\qed

\section*{Exercise 3}
\subsection*{a)}
\newcommand\sqt{\widetilde\sqsubseteq}
\subsubsection*{Anti-Symmetry}
To show: $f \sqt g \land g \sqt f \implies f = g$ where $f = g \Leftrightarrow \forall d \in D: f(d) = g(d)$\\
Given: $f \sqt g \land g \sqt f$
$\Leftrightarrow \forall d \in D: f(d) \sqle g(d) \land g(d) \sqle f(d)$ \\
from ant-symmetry of $\sqle$ we get $ \forall d \in D: f(d) = g(d) \implies f = g$

\subsubsection*{Reflexivity}
To show: $f \sqt f$\\
$f \sqt f \Leftrightarrow \forall d \in D: f(d) \sqle f(d)$\\
Because of reflexivity of $\sqle$ this holds.

\subsubsection*{Transitivity}
To show: $f \sqt g \land g \sqt h$ implies $f \sqt h$.\\
$f \sqt g \land g \sqt h \Leftrightarrow \forall d \in D: f(d) \sqle g(d)$ and $\forall d \in D: g(d) \sqle h(d)$.\\
Since $\sqle$ is transitive, we get that\\
$ \forall d \in D: f(d) \sqle g(d) \sqle h(d) \implies \forall d \in D: f(d) \sqle h(d)$\\
$\Rightarrow f \sqt h$
\qed

\subsection*{b)}
Let $S= \{f_1, f_2, \dots\}$ be a arbitrary chain in $(D \rightarrow D, \sqt)$.\\
$S$ induces a chain $S_d$ for each $d \in D$:
$S_d = \{f_1(d), f_2(d), \dots\} \subseteq D$\\
since $(D, \subseteq)$ is a complete lattice $\bigsqcup S_d$ exists for every $d \in D$, 
then $L: D \rightarrow D$ with $L(d) =\bigsqcup S_d$ is LUB of $S$ w.r.t. $\sqt$.
Since this holds for every chain $(D \rightarrow D, \sqt)$ is a complete lattice. \qed


\section*{Exercise 4}
\newcommand\ein{\leavevmode{\parindent=1em\indent}}
\newcommand\cst[2]{\texttt{[#1]\textsuperscript{#2}}}
\newcommand\cli[2]{\cst{#1}{#2}\texttt{;}\\}

\vbox{c = \\
\cli{y := 19}{1}
\cli{x := y + 23}{2}
\cli{z := 1}{3}
\texttt{while \cst{y < x}{4} begin}\\
\ein \cli{y := x + z}{5}
\texttt{end}\\
\cli{y := x * x}{6}}

\begin{align*}
	\varphi_1(a,b,c) &= (a, 19, c)\\
	\varphi_2(a,b,c) &= (b+23, b, c)\\
	\varphi_3(a,b,c) &= (a, b, 1)\\
	\varphi_4(a,b,c) &= (a, b, c)\\
	\varphi_5(a,b,c) &= (a, a+c, c)\\
	\varphi_6(a,b,c) &= (a, a*a, c)\\
\end{align*}

%\begin{align*}
	%mop(6) = \varphi_6(\varphi_5(\varphi_4(\varphi_3(\varphi_2(\varphi_1(\top,\top,\top))))))
%\end{align*}

\begin{align*}
&Path(1) = \{[1]\}\\
&Path(2) = \{[1,2]\}\\
&Path(3) = \{[1,2,3]\}\\
&Path(4) = \{[1,2,3,4],[1,2,3,4,5,4],..\}\\
&Path(5) = \{[1,2,3,4,5],[1,2,3,4,5,4,5],..\}\\
&Path(6) = \{[1,2,3,4,6],[1,2,3,4,5,6],[1,2,3,4,5,4,6],..\}\\
\end{align*}

\begin{align*}
	mop(1)&=\varphi_{[1]}(\top,\top,\top)=(\top, 19, \top))\\
	mop(2)&=\varphi_{[1,2]}(\top,\top,\top)=(42, 19, \top)\\
	mop(3)&=\varphi_{[1,2,3]}(\top,\top,\top)=(42, 19, 1)\\
	mop(4)&=\varphi_{[1,2,3,4]}(\top,\top,\top) \sqcup \varphi_{[1,2,3,4,5,4]}(\top,\top,\top) \sqcup .. \\
	&= (42,19,1) \sqcup (42,43,1) \sqcup .. = (42,\top,1)\\
	mop(5)&=\varphi_{[1,2,3,4,5]}(\top,\top,\top) \sqcup \varphi_{[1,2,3,4,5,4,5]}(\top,\top,\top) \sqcup .. \\
	&=(42, 43, 1) \sqcup (42,43,1) \sqcup .. = (42,43,1)\\
	mop(6)&=\varphi_{[1,2,3,4,6]}(\top,\top,\top) \sqcup \varphi_{[1,2,3,4,5,6]}(\top,\top,\top) \sqcup \varphi_{[1,2,3,4,5,4,6]}(\top,\top,\top)\\
	&=(42, 1764, 1) \sqcup (42,1764,1) \sqcup (42,1764,1) \sqcup .. = (42,1764,1)\\
\end{align*}
\end{document}
